\chapter{绪论}

课题来源与研究内容

\section{目的与意义}

农业是国民经济组成的重要部分,农作物生产对于社会的稳定具有重要作用。农作物产量是政府部门进行农业决策和宏观调控的重要依据,预估农作物产量具有重要的意义。影响农作物产量的因素众多,农作物产量的形成通常具有非线性的特点,准确地预估农作物产量一直是农业发展中的一个难题。本文设计了一种基于深度学习的算法针对农作物产量预测问题进行求解。

我国各地自然条件、社会经济条件差异明显,农业生产水平差据较大,农业集约化总体水平较低。根据表\ref{tab:intro1},我国农业具有以下特点:一、农业人口人均耕地面积小,仅为世界平均水平的1/5;低于印度,甚至低于国土面积小的日本,同美国更是相差甚远。二、农业机械化水平低。每万公顷拖拉机拥有量,仅为世界平均水平,甚至低于印度。三、化肥投入水平高。每公顷化肥投入量是世界平均水平的3.37倍,高于美国、日本等发达国家。 

\begin{table}[htbp]
    \centering
    \caption{中美印日农业集约化程度比较(2005年)}
    \label{tab:intro1}
    \begin{tabular}{cccc}
      \toprule
        国家 &  农业人口平均耕地面积 & 拖拉机配备量 & 化肥使用量\\
      \midrule
        中国 & 0.3 & 7.1 & 341.0 \\
        印度 & 0.6 & 15.9 & 129.0 \\
        日本 & 2.1 & 461.2 & 269.8 \\
        美国 & 63.7 & 3.7 & 110.5 \\
      \midrule
        世界平均 & 1.5 & 8.9 & 121.7 \\
      \bottomrule
    \end{tabular}
    \note{数据来自《世界统计年鉴-2005》}
  \end{table}

\subsection{二级节标题}

\subsubsection{三级节标题}

\paragraph{四级节标题}

\subparagraph{五级节标题}

\section{脚注}

Lorem ipsum dolor sit amet, consectetur adipiscing elit, sed do eiusmod tempor
incididunt ut labore et dolore magna aliqua. Ut enim ad minim veniam, quis
nostrud exercitation ullamco laboris nisi ut aliquip ex ea commodo consequat.
Duis aute irure dolor in reprehenderit in voluptate velit esse cillum dolore eu
fugiat nulla pariatur. Excepteur sint occaecat cupidatat non proident, sunt in
culpa qui officia deserunt mollit anim id est 
\footnote{This is a long long long long long long long long long long long long
long long long long long long long long long long footnote.}

\section{国内外文献综述}


\section{研究方法与思路}
